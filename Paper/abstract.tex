% abstract

There is an often acknowledged lack of reference databases in tread pattern forensics which prevents formalization of the scientific principles underlying the forensic analysis of footwear and tire tread marks. In this paper, we describe a longitudinal study of 160 pairs of athletic shoes, and the corresponding database which makes the data from the study available to the community. In addition to measuring shoe wear at multiple points over an approximately 6 month period, the study used six different recording methods to capture data about the condition of the athletic shoes: 2D digital scans, 3D scans, fingerprint powder and adhesive film, fingerprint powder and paper, digital photography, and fingerprint powder on vinyl flooring. In addition to these measures, we also recorded the number of steps taken in the shoes using a pedometer, the surfaces the shoes were used on, and other auxiliary information. As an additional indicator of likely wear pattern formation, at the beginning of the study, we recorded the weight distribution of each participant (static and dynamic) using a pressure mat scanner. All of this information is made available in an online database for the community to use to investigate pressing questions about shoe wear patterns, comparisons between different capture methods, RAC formation, and more.
